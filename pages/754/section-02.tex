%%%%this file inputes all section-02-*.tex files. 
\chapter{Overview of various smoothers}
A scatter plot smoother is a tool for finding structure in a scatter
plot: $(x_1,y_1),\dots,(x_n,y_n)$


\begin{figure}[htp]
\caption{\label{f2.1} CD4 cell count since seroconversion for HIV
  infected men.} 
\centerline{\epsfig{figure=Plots/plot-02-01.ps,angle=270,width=.8\textwidth}}
\end{figure}

\begin{itemize}
\item Suppose that we consider $\by = (y_1,\dots,y_n)'$ as the {\it response
measurements} and $\bx = (x_1,\dots,x_n)'$ as the {\it design points}.

\item We can think of $\bx$ and $\by$ as outcomes of random variable $X$
and $Y$. However, for scatter plot smoothers we don't
really need stochastic assumptions, it can be considered as a
descriptive tool.

\item A scatter plot smoother can be defined as a function (remember
  the general definition of {\it function}) of $\bx$ and
$\by$ with domain at least containing the values in $\bx$: $s =
\bS[\by | \bx]$. 
 
\item There is usually a ``recipe'' that gives $s(x_0)$, which is the function
$\bS[\by | \bx]$ evaluated at $x_0$,  for all $x_0$. We will be
calling $x_0$ the {\it target value} when we giving the recipe. Note:
Some recipes don't give an $s(x_0)$ for all $x_0$, but only for the
$x$'s included in $\bx$. 

\end{itemize}

Note we will call the vector $\{s(x_1),\dots,s(x_n)\}'$ as {\it the smooth}.

Here is a stupid example: If we assume a random desing model and take
expectations over the empirical distribution $\hat{F}$, defined by the
observations, we have for any $x_0 \in \{x_1,\dots,x_n\}$, 
\[
E_{\hat{F}}[Y|X=x_0] = \ave \{ y_i; x_i = x_0\}. 
\]
Define $s(x_0)=\E_{\hat{F}}[Y|X=x_0]$. What happens if the $x_i$ are
unique? 

Since $Y$ and $X$ are, in general,  non-categorical we don't expect to
find many replicates at any given value of $X$. This means that we
could end up with the data again, $s(x_0)=y_0$ for all $x_0$. Not very
smooth!

Note: For convenience, through out this chapter,  we assume that the
data are sorted by $X$. 

Many smoothers force  $s(x)$ to be a smooth function of $x$. This is
a fancy way of saying we think data points that are close (in $x$)
should have roughly the same expectation. 

\newpage

\section{Parametric smoother}
These are what you have seen already. We force a function defined by
``few'' parameters on the data and use something like least squares to
find the ``best'' estimates for the parameters. 

For example, a regression line computed with least squares can be
thought of as a smoother. In this case $S[\by|\bx](x_0) =
(1 \, x_0) \, (\bX'\bX)^{-1}\bX'\by$ with $\bX$ a design matrix containing a
column of 1's and $\bx$ ({\tt cbind(1,x)}).
 

The lack of flexibility of these types of smoother can make them
provide misleading results.


\begin{figure}[htp]
\caption{\label{f2.2} CD4 cell count since seroconversion for HIV infected men.}
\centerline{\epsfig{figure=Plots/plot-02-02.ps,angle=270,width=.8\textwidth}}
\end{figure}


\newpage

\section{Bin smoothers}
A bin smoother, also known as a regressogram, mimics a categorical
smoother by partitioning the predicted value into disjoint and
exhaustive regions, then averaging the response in each
region. Formally, we choose cut-points $c_0 < \dots <c_K$ where $c_0 =
-\infty$ and $c_K = \infty$, and define
\[
R_k = \{i; c_k \leq x_i < c_{k+1} \}; k=0,\dots,K
\]
the indexes of the data points in each region. Then $S[\by|\bx]$ is
given by 
\[
s(x_0) = \ave_{i \in R_k}\{ y_i \} \mbox{ if } x_0 \in R_k
\]

Notice that the bin smoother will have discontinuities. 


\begin{figure}[htp]
\caption{\label{f2.3} CD4 cell count since seroconversion for HIV
  infected men.} 
\centerline{\epsfig{figure=Plots/plot-02-03.ps,angle=270,width=.8\textwidth}}
\end{figure}



\newpage

\section{Running-mean/moving average}
Since we have no replicates and we want to force $s(x)$ to be smooth
we can use the motivation that under some stastical model, for any
$x_0$ values of $f(x)=\E[Y|X=x]$ for 
$x$ close to $x_0$ are similar. 

How do we define close? 
A formal definition is the {\it symmetric nearest neighborhood}
\[
N^S(x_i) = \{ \max(i-k,1),\dots,i-1,i,i+1,\min(i+k,n) \}
\]

We may now define running mean as:
\[
s(x_i) = \ave_{j \in N^S(x_i)} \{ y_j \}
\]

We can also forget about the symmetric part and simply define the
nearest $k$ neighbors. 


\begin{figure}[htp]
\caption{\label{f2.4} CD4 cell count since seroconversion for HIV infected men.}
\centerline{\epsfig{figure=Plots/plot-02-04.ps,angle=270,width=.8\textwidth}}
\end{figure}

This usually too wiggly to be considered useful.  Why do you think?

Notice we can also fit a line instead of a constant. This procedure is
called running-line.

Can you write out the recipe for $s(x_i)$ for the running-line smoother?






\section{Kernel smoothers}
One of the reasons why the previous smoothers is wiggly is because
when we move from $x_i$ to $x_{i+1}$ two points are usually changed in
the group we average. If the new two points
are very different then  $s(x_i)$ and $s(x_{i+1})$ may be quite
different. One way to try and fix this is by making the transition
smoother. That's the idea behind kernel smoothers.

Generally speaking a kernel smoother defines a set of weights
$\{W_i(x)\}_{i=1}^{n}$ for each $x$ and defines 
\[
s(x) = \sum_{i=1}^n W_i(x) y_i.
\]

We will see that most scatter plot smoothers can be considered to be
kernel smoothers in this very general definition. 

What is called a kernel smoother in practice has a simple approach to
represent the weight sequence $\{W_i(x)\}_{i=1}^{n}$ by describing the
shape of the weight function $W_i(x)$ by a density function with a
scale parameter that adjusts the size and the form of the weights near
$x$. It is common to refer to this shape function as a {\it kernel}
$K$. The kernel is a continuous, bounded, and symmetric real function
$K$ which integrates to one,

\[
\int K(u)\,du = 1.
\]

For a given scale parameter $h$, the weight sequence is then defined
by
\[
W_{hi}(x) = \frac{K\left( \frac{x - x_i}{h} \right) }{ \sum_{i=1}^n K\left
  ( \frac{ x - x_i }{h} \right)}
\]

Notice: $\sum_{i=1}^n W_{hi} (x_i) = 1$


The kernel smoother is then defined for any $x$ as before by
\[
s(x) = \sum_{i=1}^n W_{hi}(x) Y_i.
\]

Notice: if we consider $x$ and $y$ to be observations of random
variables $X$ and $Y$ then one can get an intuition for why this would
work because
\[
E[ Y | X ] = \int y f_{X,Y}(x,y) \, dy / f_X(x),
\]
with $f_X(x)$ the marginal distribution of $X$ and $f_{X,Y}(x,y)$ the joint
distribution of $(X,Y)$, and 
\[
s(x) = \frac{ n^{-1}\sum_{i=1}^n K\left( \frac{x - x_i}{h} \right) y_i }
  { n^{-1}\sum_{i=1}^n K\left   ( \frac{ x - x_i }{h} \right)}
\]

Because we think points that are close together are similar, a kernel
smoother usually defines weights that decrease in  
a smooth fashion as one moves away from the target point. 

Running mean smoothers are kernel smoothers that use a ``box'' kernel. A
natural candidate for $K$ is the standard Gaussian density. (This is 
very inconvenient computationally because its never 0).  This smooth
is shown in Figure \ref{f2.5} for $h=1$ year.


\begin{figure}[htp]
\caption{\label{f2.5} CD4 cell count since seroconversion for HIV infected men.}
\centerline{\epsfig{figure=Plots/plot-02-05.ps,angle=270,width=.8\textwidth}}
\end{figure}

In Figure \ref{f2.6} we can see the weight sequence for the box and Gaussian
kernels for three values of $x$.

\begin{figure}[htp]
\caption{\label{f2.6} CD4 cell count since seroconversion for HIV infected men.}
\centerline{\epsfig{figure=Plots/plot-02-06.ps,angle=270,width=.8\textwidth}}
\end{figure}


\newpage

\subsection{An Asymptotic result}
For the asymptotic theory presneted here we will assume the stochastic
design model with a one-dimensional covariate. 

For the first time in this Chapter we will set down a specific
stochastic model. Assume we have $n$ IID observations of the random
variables $(X,Y)$ 
and that  
\begin{equation}
\label{simplemodel}
Y_i = f(X_i) + \varepsilon_i, i=1,\dots,n
\end{equation}
where $X$ has marginal distribution $f_X(x)$ and the $\varepsilon_i$ IID
errors independent of the $X$. A common extra assumption is that the
errors are normally distributed.
We are now going to
let $n$ go to infinity... What does that mean?

For each $n$ we define an estimate for $f(x)$ using the kernel
smoother with scale parameter $h_n$.

\begin{theorem}
\label{t2.1}
Under the following assumptions
\begin{enumerate}
\item $\int |K(u)| \, du < \infty$
\item $\lim_{|u| \rightarrow \infty} uK(u) = 0$
\item $\E(Y^2) \leq \infty$
\item $n \rightarrow \infty, h_n \rightarrow 0, nh_n \rightarrow
  \infty$
\end{enumerate}

Then, at every point of continuity of $f(x)$ and  $f_X(x)$ we have
\[
 \frac{ \sum_{i=1}^n K\left( \frac{x - x_i}{h} \right) y_i }
  { \sum_{i=1}^n K\left   ( \frac{ x - x_i }{h} \right)}
  \rightarrow f(x) \mbox{ in probability.}
\]
\end{theorem}

{\bf Proof:} Homework. Hint: Start by proving the fixed design model.



\section{Linear smoothers}
Most of the smoother presented here are linear smoothers which means
that the fit at any point $x_0$ can be written as 
\[
s(x) = \sum_{j=1}^n S_{j}(x) y_j.
\]

In practice we usually have the model 
\[
Y_i = f(X_i) + \epsilon_i
\]
and we have observations $\{(x_i, y_i)\}$. Many times it is the vector
${\mathbf f} = \{f(x_1),\dots,f(x_n)\}'$ we are after. In this case
the vector of estimates $\hat{\mathbf f} =
\{\hat{f}(x_1),\dots,\hat{f}(x_n)\}'$ can be written as
\[
\hat{\mathbf f} = \bS \by
\]
with $\bS$ a matrix with the i,j-th entry $S_{j}(x_i)$. We will call
$\hat{\mathbf f}$ the {\it smooth}.

This makes it easy to figure out things like the variance of
$\hat{\f}$ since 
\[
\var[\bS \by] = \bS \var[\by] \bS' 
\]
which in the case of IID data is $\sigma^2 \bS\bS'$.





