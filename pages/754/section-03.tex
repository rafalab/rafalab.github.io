%%%this is taken form loader (1999)

\chapter{Local Regression}
Local regression is used to model a relation between a predictor
variable and response variable. To keep things simple we will consider
the fixed design model. We assume a model of the form
\[
Y_i = f(x_i) + \varepsilon_i
\]
where $f(x)$ is an unknown function and $\varepsilon_i$ is an error term,
representing random errors in the observations or variability from
sources not included in the $x_i$.

We assume the errors $\varepsilon_i$ are IID with mean 0 and finite
variance $\var(\varepsilon_i) = \sigma^2$. 

We make no global assumptions about the function $f$ but assume that
locally it can be well approximated with a member of a simple class of
parametric function, e.g. a constant or straight line. Taylor's
theorem says that any continuous function can be approximated with
polynomial. 
 
\section{Taylor's theorem}
We are going to show three forms of Taylor's theorem. 
\begin{itemize}

\item This is the original. Suppose $f$ is a real function on $[a,b]$, $f^{(K-1)}$ is continuous on
$[a,b]$,  $f^{(K)}(t)$ is bounded for $t \in (a,b)$ then
for any distinct points $x_0 < x_1$ in $[a,b]$ there exist a point
$x$ between $x_0 < x < x_1$ such that 
\[
f(x_1) = f(x_0) + \sum_{k=1}^{K-1} \frac{f^{(k)}(x_0)}{k!}(x_1-x_0)^k +
\frac{f^{(K)}(x)}{K!}(x_1 - x_0)^K.
\]
{\bf Notice:} if we view $f(x_0) + \sum_{k=1}^{K-1}
\frac{f^{(k)}(x_0)}{k!}(x_1-x_0)^k$ as function of $x_1$, it's a
polynomial in the family of polynomials  
\[
{\cal  P}_{K+1}= \{f(x) = a_0 + a_1 x + \dots + a_K x^K,
(a_0,\dots,a_K)' \in {\mathbb R}^{K+1}\}.
\]

\item Statistician sometimes use what is called Young's form of Taylor's
Theorem:

Let $f$ be such that $f^{(K)}(x_0)$ is bounded for $x_0$ then
\[
f(x) = f(x_0) +  \sum_{k=1}^{K} \frac{f^{(k)}(x_0)}{k!}(x-x_0)^k +
o(|x-x_0|^K), \mbox{ as } |x-x_0| \rightarrow 0.
\]
{\bf Notice:} again the first two term of the right hand side is in ${\cal P}_{K+1}$.


\item In some of the asymptotic theory presented in this class we are
  going to use another refinement of Taylor's theorem called Jackson's
  Inequality: 

  Suppose $f$ is a real function on $[a,b]$ with $K$ is continuous
  derivatives then
  \[
  \min_{g \in {\cal P}_k} \sup_{x \in [a,b]} |g(x) - f(x)| \leq C \left(
    \frac{b-a}{2k}\right)^K 
  \]
  with ${\cal P}_k$ the linear space of polynomials of degree $k$.
\end{itemize}

\section{Fitting local polynomials}
We will now define the recipe to obtain a loess smooth for a target
covariate $x_0$. 

The first step in loess is to define a weight function (similar to the
kernel $K$ we defined for kernel smoothers). For computational
and theoretical purposes we will define this weight function so that
only values within a {\it smoothing window} $[x_0+h(x_0),x_0-h(x_0)]$ will be
considered in the estimate of $f(x_0)$.  

Notice: In local regression $h(x_0)$ is called the span or bandwidth. It
is like the kernel smoother scale parameter $h$. As will be seen a bit
later, in local regression, the span may depend on the target
covariate $x_0$.

This is easily achieved by
considering  weight functions that are $0$ outside of $[-1,1]$. For example
Tukey's tri-weight function
\[
W(u) = \left\{ \begin{array}{cc}
(1 - |u|^3)^3&|u| \leq 1\\
0&|u| > 1.
\end{array}
\right.
\]

The weight sequence is then easily defined by
\[
w_i(x_0) = W \left( \frac{x_i - x_0}{h(x)} \right)
\]

We define a window by a procedure similar to the $k$ nearest
points. We want to include $\alpha\times 100$\% of the data. 

Within the smoothing window, $f(x)$ is approximated by a
polynomial. For example, a quadratic approximation
\[
f(x) \approx \beta_0 + \beta_1 (x-x_0) + \frac{1}{2} \beta_2 (x-x_0)^2 \mbox{ for
  } x \in [x_0 - h(x_0), x_0+h(x_0)].
\]
For continuous function, Taylor's theorem tells us something about how
good an approximation this is.

To obtain the local regression estimate $\hat{f}(x_0)$ we simply find
the $\bb = (\beta_0,\beta_1,\beta_2)'$ that minimizes
\[
\hat{\bb} = \arg \min_{\bb \in {\mathbb R}^3} \sum_{i=1}^n w_i(x_0)[ Y_i - \{\beta_0 + \beta_1 (x_i-x_0) + \frac{1}{2} \beta_2 (x_i-x_0)\}]^2
\]
and define $\hat{f}(x_0) = \hat{\beta}_0$.

Notice that the Kernel smoother is a special case of local
regression. Proving this is a Homework problem.

\section{Defining the span}
In practice, it is quite common to have the $x_i$ irregularly
spaced. If we have a fixed span $h$ then one may have local estimates
based on many points and others is very few. For this reason we may
want to consider a nearest neighbor strategy to define a span for
each target covariate $x_0$.

Define $\Delta_i(x_0) = |x_0 -
x_i|$, let $\Delta_{(i)}(x_0)$ be the ordered values of such
distances. One of the arguments in the local regression function {\tt
  loess()} (available in the modreg library) is the {\tt span}. A span
of $\alpha$ means that for each local fit we want to use
 $\alpha \times 100 \%$ of the data.  

Let $q$ be equal to $\alpha$n
truncated to an integer. Then we define the span $h(x_0) =
\Delta_{(q)}(x_0)$. As $\alpha$ increases the estimate
becomes smoother. 

In Figures \ref{f3.1} -- \ref{f3.3} we see loess smooths for the CD4
cell count data using spans of
0.05, 0.25, 0.75, and 0.95. The smooth presented in the Figures are
fitting a constant, line, and parabola
respectively.


\begin{figure}[htp]
\caption{\label{f3.1} CD4 cell count since seroconversion for HIV infected men.}
\centerline{\epsfig{figure=Plots/plot-03-01.ps,width=\textwidth}}
\end{figure}

\begin{figure}[htp]
\caption{\label{f3.2} CD4 cell count since seroconversion for HIV infected men.}
\centerline{\epsfig{figure=Plots/plot-03-02.ps,width=\textwidth}}
\end{figure}

\begin{figure}[htp]
\caption{\label{f3.3} CD4 cell count since seroconversion for HIV infected men.}
\centerline{\epsfig{figure=Plots/plot-03-03.ps,width=\textwidth}}
\end{figure}


\newpage 

\section{Symmetric errors and Robust fitting}
If the errors have a symmetric distribution (with long tails), or if
there appears to be 
outliers we can use robust loess.


We begin with the estimate described above $\hat{f}(x)$. The residuals
\[
\hat{\varepsilon}_i = y_i  - \hat{f}(x_i)
\]
are computed.

Let
\[
B(u;b) = \left\{ \begin{array}{cc}
\{1 - (u/b)^2\}^2&|u|<b\\
0& |u|\geq b
\end{array}
\right.
\]
be the bisquare weight function. Let $m$ = median($|\hat{\varepsilon}_i|$).
The robust weights are
\[
r_i = B(\hat{\varepsilon_i}; 6m)
\]
The local regression is repeated but with new weights $r_i w_i(x)$. The
robust estimate is the result of repeating the procedure several times.

If we believe the variance $\var(\varepsilon_i) = a_i \sigma^2$ we could
also use this double-weight procedure with $r_i = 1/a_i$.

\subsection{Example}

Radiolabeling based gene expression measurements are useful for cancer
research because they can be carried out using small amounts of
biological materials.  
Statistical issues are different from fluorescence
expression data, because radiolabeling gives absolute intensities that
reflect gene expression and 
there is no internal control. 

The data-set described here was obtained to identify genes that
may be associated with lung cancer. Lung cancer tissue was obtained
from various subjects. Normal tissues from the same type of cells was
obtained from those same subjects. From each of   
these tissues 2 samples were prepared using 2 different isotopic
batches. Each of these 4 samples were hybridized with a filter
spotted with cDNA from many genes in a $48 \times 24$ grid. We refer
to these spotted filters as arrays. Each of these arrays were scanned to
produce an image file 
which was then analyzed with 
specialized software that produced an intensity level for each grid
point or {\it spot} on the array. 

Not all the values read  from the arrays are associated with
genes. There were 207 spots where
no cDNA was spotted. They were left empty. Because there is {\it
  non-specific} binding between the samples and the filters, positive
values are 
obtained from these empty spots. The intensities read 
from these empty spots provide direct evidence about measurement error
associated with the system. Spots associated with genes that are not
expressed will also have intensities due to non-specific binding.


Can we rank genes by differential expression between
cancer and normal tissues in each subject? 

If we denote with $\bx$ and $\by$ the log intensities of each spot we
could say a gene is differentially expressed if $\by - \bx$ is
significantly bigger than 0 for the spot related to that gene.
One problem with this is that there is a filter effect, so $\by$ can
be systematically smaller than $\bx$.

A common procedure in microarray data analysis is to simply normalize the
filters by subtracting the mean of each filter from each value,
i.e. consider $y^{(normalized)}_{i} =  y_{i} -
\bar{y}$ and similarly for the $x$s. The danger with doing
this is that many of the genes spotted on the arrays are usually
selected because researchers consider them likely to be
over-expressed. This means 
that the mean of the $y$s should be larger than the $x$s and this
difference in mean is confounded with the difference in filter
effect. By subtracting means we would be subtracting out some of the
differential expression between cancer and normal
tissues. 

In Figure \ref{f3.4} we plot the ratio of the intensities vs. the
product of the 
intensities in a log scale, i.e. $y-x$ vs. $x + y$, for the two
replicates of subject 1. Notice that the
{\it filter effect} seems to change with the total intensity of a
particular spot. For this reason using medians or trimmed
means to remove the filter effect is not a good solution. If we model $x$
and $y$ as random 
variables then we have that the expected filter effect depends on the
total intensity, i.e. $\mbox{E}(y - x | x+y )$ is not constant.
This arises because
specific binding and non-specific binding are two different natural
processes. Because we have no way of knowing which points represent
non-specific binding and which represent specific binding we cannot
normalize by just estimating two means. Rather, we estimate
$\mbox{E}(y-x|y+x)$ using loess. It is critical to use a robust loess,
so that large differences do not affect the fit too much. Notice in
Figure \ref{f3.4} the difference in the robust and non-robust
estimates.


\begin{figure}[htp]
\caption{\label{f3.4} Total intensity plotted against ratio with a
  loess prediction using Gaussian and symmetric kernel.}
\centerline{\epsfig{figure=Plots/plot-03-04.ps,angle=270,width=.8\textwidth}}
\end{figure}




\section{Multivariate Local Regression}
Because Taylor's theorems also applies to multidimensional functions it
is relatively straight forward to extend local regression to cases
where we have more than one covariate. For example if we have a
regression model for two covariates
\[
Y_i = f(x_{i1},x_{i2}) + \varepsilon_i
\]
with $f(x,y)$ unknown. Around a target point $\bx_0 = (x_{01},x_{02})$
a  local quadratic approximation is now 
\[
f(x_1,x_2) \approx \beta_0 + \beta_1 (x_1 - x_{01}) + \beta_2 (x_2 - x_{02})
+ \beta_3 (x_1 - x_{01})(x_2 - x_{02}) + \frac{1}{2} \beta_4 (x_1 -
x_{01})^2 +  \frac{1}{2} \beta_5(x_2 -
x_{02})^2  
\]

Once we define a distance, between a point $\bx$ and
$\bx_0$, and a span $h$ we can define define waits as in the previous
sections:
\[
w_i(\bx_0) = W\left(\frac{||\bx_i,\bx_0||}{h}\right).
\]
It makes sense to re-scale $x_1$ and $x_2$ so we smooth the same way
in both directions. This can be done through the distance function,
for example by defining a distance for the space ${\mathbb R}^d$ with
\[
||\bx ||^2 = \sum_{j=1}^d (x_j/v_j)^2
\]
with $v_j$ a scale for dimension $j$. A natural choice for these $v_j$
are the standard deviation of the covariates.

Notice: We have not talked about k-nearest neighbors. As we will see in
Chapter VII the {\it curse of dimensionality} will make this hard.

\subsection{Example}
We look at part of the data obtained from a study by Socket
et. al. (1987) on
the factors affecting patterns of insulin-dependent diabetes mellitus
in children. The objective was to investigate the dependence of the
level of serum C-peptide on various other factors in order to
understand the patterns of residual insulin secretion. The response
measurement is the logarithm of C-peptide concentration (pmol/ml) at
diagnosis, and the predictors are age and base deficit, a measure of
acidity. In Figure \ref{f3.5} we show a loess two dimensional
smooth. Notice that the effect of age is clearly non-linear.

\begin{figure}[htp]
\caption{\label{f3.5} Loess fit for predicting C.Peptide from  Base.deficit and Age.}
\centerline{\epsfig{figure=Plots/plot-03-05.ps,width=.7\textwidth}}
\end{figure}


\begin{thebibliography}{}


\bibitem{1}
{Cleveland, R.~B., Cleveland, W.~S., McRae, J.~E., and Terpenning, I.} (1990).
\newblock Stl: {A} seasonal-trend decomposition procedure based on loess.
\newblock {\em Journal of Official Statistics}, 6:3--33.

\bibitem{2}
{Cleveland, W.~S. and Devlin, S.~J.} (1988).
\newblock Locally weighted regression: {A}n approach to regression analysis by
  local fitting.
\newblock {\em Journal of the American Statistical Association}, 83:596--610.

\bibitem{3}
{Cleveland, W.~S., Grosse, E., and Shyu, W.~M.} (1993).
\newblock Local regression models.
\newblock In {Chambers, J.~M. and Hastie, T.~J.}, editors, {\em Statistical
  Models in {S}}, chapter~8, pages 309--376. Chapman \& Hall, New York.

\bibitem{4} 
{Loader, C.~R.}  (1999),
 {\it Local Regression and Likelihood}, New York:
 Springer.

\bibitem{5}
{Socket, E.B., Daneman, D. Clarson, C., and Ehrich,
  R.M.} (1987). Factors affecting and patterns of residual insulin
  secretion during the first year of type I (insulin dependent) diabetes
  mellitus in children. {\it Diabetes} 30, 453--459.

\end{thebibliography}
