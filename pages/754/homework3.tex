\documentclass[12pt]{article}
\usepackage{times,amsfonts,my-commands}

\begin{document}

\begin{center}
{\bf Homework 3: Due May 8}
\end{center}

\begin{enumerate}

\item Suppose $\bB$ is an $n \times (K+4)$ matrix containing the
  evaluation of the cubic B-splines basis functions with $K$ interior
  knots evaluated at the $N$ values of $X$. Let $\bC$ be the $2 \times
  (K+4)$ matrix containing the second derivatives of the basis
  function at the boundary points $x_1$ and $x_n$. Show how to derive
  $\bN$ from $\bB$, an $n \times (K+2)$ basis matrix for the natural
  cubic splines with the same interior knots and boundary knots at
  the extremes of $X$. First do the math and then describe how you
  would do this using R. 

\item Write a one page summary describing your final project. Describe
  the data set, the scientific question, its significance, and a brief
  summary of the analyses you plan to carry out. 

\end{enumerate}

\end{document}



