\documentclass[12pt]{article}
\usepackage{epsfig,amsfonts,my-commands,times}
\title{Homework 2}
\date{Due: April 19}
\begin{document}


\begin{center}
{\bf Homework 2: Linear Smoothers\\
Due April 19th\\ PART I}\end{center}

\begin{enumerate}
\item The Strontium data set was collected to test several hypotheses 
about the catastrophic events that occurred approximately 65 million
years ago. The data contains Age in million of years and the ratios
described here. There is a division between
two geological time periods, the Cretaceous (from 66.4 to 144 million
years ago) and the 
Tertiary (spanning from about 1.6 to 66.4 million years ago). Earth
scientist believe that the boundary between these periods is
distinguished by tremendous changes in climate that accompanied a mass
extension of over half of the species inhabiting the planet at the
time. Recently, the compositions of Strontium (Sr) isotopes in sea
water has been used to evaluate several hypotheses about the cause of
these extreme events. The dependent variable of the data-set is related to
the isotopic make up of Sr measured for the shells of marine
organisms. The Cretaceous-Tertiary boundary is referred to as
KTB. The   data show a 
peak  at this time and this is used as 
evidence that a meteor collided with earth. 

The data presented in the figure represents standardized ratio of
strontium-87 isotopes ($^{87}$Sr) to strontium-86 isotopes ($^{86}$Sr)
contained in the shells of foraminifera fossils taken form cores
collected by deep sea drilling. For each sample its time in history is
computed and  the standardized ratio is computed:
\[
^{87}\delta \mbox{Sr} = \left( \frac{ ^{87}\mbox{Sr}/^{86}\mbox{Sr sample}}{^{87}\mbox{Sr}/^{86}\mbox{Sr sea
    water}} - 1\right) \times 10^5.
\] 
Earth scientist expect that $^{87}\delta \mbox{Sr}$ is a smooth-varying
function of time and that deviations from smoothness are  mostly
measurement error. 

Consider the model 
\[
y = f(x) + \epsilon, x \in I=[a,b] \subset \mathbb{R}.
\]
and assume that the $\epsilon$'s are IID $N(0,\sigma^2)$. 

\begin{enumerate}
\item Consider the least squares estimates obtained when we assume
  that $f \in \cal  G$, the space of polynomials of order
  4, 6, 8 and 12. Discuss  the estimates obtained under these
  assumptions. (Hint: talk about 
  degrees of freedom, over-fitting and under-fitting)

\item Given a sequence $a = t_0 < t_1 < \dots < t_m < t_{m+1} = b$,
  construct $m+1$ (disjoint) intervals
\[
I_l = [t_{l-1},t_l), 1 \leq l \leq m \mbox{ and } I_{m+1} =
[t_m,t_{m+1}],
\]
whose union is $I=[a,b]$. Define the piecewise polynomials of order
$k$  
\[
g(x) = \left\{   \begin{array}{cc}
    g_1(x) = \theta_{1,1} + \theta_{1,2} x + \dots + \theta_{1,k}
    x^{k-1},&x \in I_1\\
    \vdots&\vdots\\
    g_{m+1}(x) = \theta_{m+1,1} + \theta_{m+1,2} x + \dots + \theta_{m+1,k}
    x^{k-1},&x \in I_{k+1}.
\end{array}
\right.
\]
Is the space containing these function a linear space? If so write out
a basis matrix and explain how you would estimate the parameters
$\bg{\theta}=(\theta_{1,1},\dots,\theta_{m+1,k})'$ if we were to
assume that $f$ is in this space.

\item In the previous exercise are there any conditions you need for the
  $\{X_i\}$'s and $I_m$'s for the solution to be unique?

\item 
Try fitting the following piecewise polynomials (you choose the
  break-points)
\begin{enumerate}
\item k = 1, 2-pieces,
\item k = 2, 2-pieces,
\item k = 1, 4-pieces, and 
\item k = 2, 4-pieces. 
\end{enumerate}
Comment on the fits and compare them to the ones obtained in
exercise 2 in Part I.

\item Construct a test to see if the two quadratic
  pieces in pard (d) part ii are the same. Explain why this test doesn't really make 
  sense in practice.

\end{enumerate}

\item Find an estimate of the regression function $f$ defined for the
  Strontium data set using natural smoothing
  splines. Use cross-validation to choose a penalty coefficient
  $\lambda$. Make a plot of
  CV($\lambda$) vs. $\lambda$ and compare the ``smooths''.
\item Use loess to estimate the regression function $f$ defined for
  the Strontium data set. 
\begin{enumerate}
\item Use cross-validation to choose the smoothing
  parameter (what R calls {\tt span}).
\item Assume the errors are normal and construct point-wise confidence
  intervals make a plot.
\end{enumerate}
\item Repeat exercise problem 2 for the CD4 cell count data. Discuss
  why the regression model used in problem 2 isn't as useful for this
  case. How does this affect the interpretation of the results.
\item Derive the following formulas for a linear smoother estimate $\hat{\f}_{\lambda}=\bS_{\lambda}\by$:
\begin{enumerate}
\item
\begin{eqnarray*}
\mbox{MSE}(\lambda) &=& n^{-1} \sum_{i=1}^n
\var\{\hat{f}_{\lambda}(x_i)\} + \ave ( \bv_{\lambda}^2 ) \\
&=& n^{-1}\tr(\bS_{\lambda}\bS_{\lambda}') \sigma^2 + n^{-1}
\bv_{\lambda}'\bv_{\lambda}\\
\mbox{PSE}(\lambda) &=& \{1 + n^{-1} tr(\bS_{\lambda}\bS_{\lambda}')
\} \sigma^2 +  n^{-1}
\bv_{\lambda}'\bv_{\lambda}.
\end{eqnarray*}
\item
\[
\E\{\mbox{ASR}(\lambda)\} = \left\{ 1 - n^{-1}\tr(2\bS_{\lambda} -
    \bS_{\lambda} \bS_{\lambda}') \right\} \sigma^2 + n^{-1}
    \bv_{\lambda}'\bv_{\lambda}
\]
\item 
\[
\mbox{CV}(\lambda) = n^{-1} \sum_{i=1}^n  \left\{ \frac{ y_i -
  \hat{f}_{\lambda}(x_i)}{1 - S_{ii}} \right\}^2
\]
\item
\[
\E[\mbox{CV}(\lambda)] \approx \mbox{PSE}(\lambda) + 2
\ave[\mbox{diag}(\bS_{\lambda}) \bv^2].
\]
\end{enumerate}
\end{enumerate}

\noindent Useful R functions: 
{\tt read.table, scan, apply, sapply, poly, cut, cbind, rep, lm, sort, order}


\end{document}


