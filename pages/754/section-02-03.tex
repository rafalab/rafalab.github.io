\section{Running-mean/moving average}
Since we have no replicates and we want to force $s(x)$ to be smooth
we can use the motivation that under some stastical model, for any
$x_0$ values of $f(x)=\E[Y|X=x]$ for 
$x$ close to $x_0$ are similar. 

How do we define close? 
A formal definition is the {\it symmetric nearest neighborhood}
\[
N^S(x_i) = \{ \max(i-k,1),\dots,i-1,i,i+1,\min(i+k,n) \}
\]

We may now define running mean as:
\[
s(x_i) = \ave_{j \in N^S(x_i)} \{ y_j \}
\]

We can also forget about the symmetric part and simply define the
nearest $k$ neighbors. 


\begin{figure}[htp]
\caption{\label{f2.4} CD4 cell count since seroconversion for HIV infected men.}
\centerline{\epsfig{figure=Plots/plot-02-04.ps,angle=270,width=.8\textwidth}}
\end{figure}

This usually too wiggly to be considered useful.  Why do you think?

Notice we can also fit a line instead of a constant. This procedure is
called running-line.

Can you write out the recipe for $s(x_i)$ for the running-line smoother?




