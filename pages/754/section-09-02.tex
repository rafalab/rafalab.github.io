\section{Information Criteria}
In this section we review the concepts behind 
Akaike's Information Criterion (AIC).

Akaike's original work is for
IID data, however it is extended to a regression type setting in a
straight forward way. Suppose that the conditional distribution of $Y$
given $\bx$ is know except for a $P$-dimensional parameter $\bb$. In
this case,  the probability density function of $\bY =
(Y_1,\dots,Y_n)$ can be written as 
\begin{equation}
\label{parsimodel}
f_{\bY}(\by;\bX,\bb) \equiv \prod_{i=1}^n f(y_i;\bx_i,\bb)
\end{equation}
with $\bX$ the design matrix with rows $\bx_i$. 

Assume that
there exists a true parameter vector $\bb^*$ defining a true
probability density denoted by
$f_{\bY}(\by;\bX,\bb^*)$. Given these assumptions, we wish to select
$\bb$, from one of the models 
defined as in (\ref{modelpdef}), ``nearest'' to the true
parameter $\bb^*$ based on the observed data $\by$.
The principle behind Akaike's criterion is to
define ``nearest'' as the model that minimizes the Kullback-Leibler 
Information Quantity
\begin{equation}
\label{KLdef} 
\Delta(\bb^*;\bX,\bb) = \int \left\{\log
  f_\bY(\by;\bX,\bb^*) - \log
  f_\bY(\by;\bX,\bb)\right\} f_{\bY}(\by;\bX,\bb^*) \, d\by.
\end{equation}

The analytical properties of the Kullback-Leibler Information Quantity
are discussed in detail by Kullback (1959) \nocite{kull:1959}. Two
important properties 
for Akaike's criterion are
\begin{enumerate}
\item $\Delta(\bb^*;\bX,\bb) > 0$ if 
$f_{\bY}(\by;\bX,\bb^*) \neq f_{\bY}(\by;\bX,\bb)$
\item
$\Delta(\bb^*;\bX,\bb) = 0$  if and only if $f_{\bY}(\by;\bX,\bb^*) =
f_{\bY}(\by;\bX,\bb)$ 
\end{enumerate}
almost everywhere on the range of $\bY$. 
The properties
mentioned suggest that finding the model that 
minimizes the Kullback-Leibler 
Information Quantity is an appropriate way to choose the ``nearest''
model.

Since the first term on the
right hand side of 
(\ref{KLdef}) is constant over all models we consider, we may instead
maximize
\begin{eqnarray}
\nonumber
H(\bb) &=& \int \log
  f_\bY(\by;\bX,\bb) f_{\bY}(\by;\bX,\bb^*) \, d\by\\
\label{Hdef}
&=& \sum_{i=1}^n \int \log f(y_i;\bX,\bb) \, f(y_i;\bx_i,\bb^*) \, dy_i.
\end{eqnarray}

Let $\bbh_p$ be the maximum likelihood estimate under
Model(p). Akaike's procedure for model selection is based on choosing
the model which produces the estimate that maximizes 
$\E_{\bb^*}\left[H(\bbh_p)\right]$ amongst all
competing models. Akaike then derives a criterion by constructing an
asymptotically unbiased estimate of $\E_{\bb^*}\left[H(\bbh_p)\right]$ based
on the observed data. 

Notice that $H(\bbh_p)$ is a function,
defined by (\ref{Hdef}),
of the maximum likelihood estimate $\bbh_p$, which is a
random variable obtained from the observed data.
A natural estimator of 
its expected value (under the true distribution of the data) is
obtained by substituting the empirical distribution of the data into
(\ref{Hdef}) 
resulting in the log likelihood equation evaluated at the maximum
likelihood estimate under model(p)
\[
l(\bbh_p) = \sum_{i=1}^n \log f(y_i;\bx_i,\bbh_p).
\]
Akaike noticed that in general  $l(\bbh_p)$ will 
overestimate $\E_{\bb^*}\left[H(\bbh)\right]$.  In particular Akaike found
that under some regularity conditions
\[
E_{\bb^*}\left[l(\bbh_p) - H(\bbh_p)\right] \approx p .
\]
This suggests that larger 
values of $p$ will result in smaller values 
of $l(\bbh_p)$, which may be
incorrectly interpreted as a ``better'' 
fit, regardless of the true model. We need to ``penalize'' for larger
values of $p$ in order to obtain an unbiased estimate of
the ``closeness'' of the model. 
This fact leads to the Akaike Information Criteria which is a
bias-corrected estimate given by 
\begin{equation}
\label{AIC}
\mbox{AIC}(p) = - 2 l(\bbh_{p}) + 2p .
\end{equation}
See, for example, Akaike (1973) and Bozdogan (1987)  for
the details.  


