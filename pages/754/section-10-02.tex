\section{Spectral Analysis}

Sometimes it is useful to describe the properties of the time series
in a frequency domain. The spectrum is defined as 
\[
f_{ab}(\lambda) = \frac{\sigma^2}{2\pi} \sum_{h = -\infty}^{\infty} \gamma_{ab}(h) \exp
( -i \lambda h)
\]

There is a one-to-one correspondence between the spectrum and the
autocovariance function 
\[
\sigma^2 \gamma_{ab}(h) = \int_{-\pi}^{\pi} f(\lambda) \exp ( i\lambda h) d\lambda
\]

We call $|f_{aa}|^2$ the power spectrum. A natural way of estimating a
power 
spectrum is using the periodogram which 
is the modulus of the Fourier transform of the data
\[
I(\lambda) = \frac{1}{2\pi T} | \sum_{t=1}^T Y_t \exp (-i \lambda t) |^2
\]
We usually compute the periodogram at the Fourier frequencies
$\lambda_j = (2 \pi j)/T, j=1,\dots,T/2$. These have desirable
statistical properties

The periodogram is also useful for detecting periodicities
(deterministic ones) in the signal. It is a mathematical fact that if
the data $Y_1,...Y_T$ has a period $p$, the the periodogram will have
peaks at frequencies $\lambda = 2 \pi T/p$ and its multiples. 

\centerline{\epsfig{figure=Plots/plot-10-04.ps,angle=270,width=\textwidth}}


\subsection{An Application}
Figured 4a and 4c shows recorded ECoG signal for two channels for
a subject that has received a sensory stimulus at some point
during the recording. A straightforward way of estimating the
spectrum of a stationary process is the periodogram
\[
I(\lambda) = \frac{1}{2\pi T} \left| \sum_t Y(t) \exp(i\lambda t)
\right|^2.\]
In Figures 4b and 4c the periodogram of this data is shown. Brain
researchers have speculated that the so-called $\alpha$ $(8-13
Hz.)$, $\beta$ $(15-25 Hz.)$, and $\gamma$ $(>30 Hz.)$ bands of
human brain signals can indicate functional activation of
sensorimotor cortex. Notice that the
periodogram exhibits a peak around frequencies 10 Hz., 20 Hz. and
60 Hz.. If we were to approximate the ECoG signal as a stationary
processes, we would describe it as having periodic components
around these frequencies. However, we are interested in learning
how the signal changes when the subjects are given a stimuli. Thus
it seems more appropriate to model the signal as a non-stationary
processes and study the time-varying spectral density.

A straightforward estimate of a time-varying spectral density
would be the dynamic periodogram. Basically, for each time $t_0$
we consider a window around that point of size $h(t_0)$ and
estimate a weighted periodogram
\[
I(t_0;\lambda) = \frac{1}{2\pi h(t_0)} \left| \sum_t
w\left(\frac{t-t_0}{h(t_0)}\right) Y(t) \exp(i\lambda t) \right|^2.
\]
Figures 4c and 4e show the estimated time-varying spectral
densities for the signals of channels 19 and 20 (lighter colors
represent higher values)
The figure seems to suggest that the $\alpha$ band changes power
and frequency after the stimulus (time 0).


\centerline{\epsfig{figure=Plots/plot-10-05.ps,width=\textwidth}}
\centerline{\epsfig{figure=Plots/plot-10-06.ps,width=\textwidth}}
\centerline{\epsfig{figure=Plots/plot-10-07.ps,width=\textwidth}}
