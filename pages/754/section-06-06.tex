\section{Mixed Models and Splines}

Mixed models are defined by 
\[
\by = \bX \bb + \bZ \bu + \beps
\]
where $\by$ is a vector of $n$ observable random variables, $\bb$
is vector of $p$ unknown parameters having fixed values (fixed
effects), $\bX$ and $\bZ$ are known matrices and $\bu$ and
$\beps$ are vector, of length $q$ and $n$, of unobservable variables
(random 
effects) such that $\E(\bu)=0$ and $\E(\beps)=0$ and 
\[
\var\left[ \begin{array}{c} \bu \\ \beps \end{array} \right] 
= 
\left[ \begin{array}{cc} G&0\\ 0&R \end{array} \right] \sigma^2
\].
Here $R$ and $G$ are known positive definite matrices and $\sigma^2$
is a positive constant.

Random effects are especially useful to introduce correlation to the
random part of the model.

Robinson (1991) describes how the best linear unbiased predictor of
$\by$ is a ``good thing''. Speed (1991) notes that after defining an
appropriate $\bX$, $\bZ$, and $G$ we have that natural smoothing are
BLUPs.

The smoothing parameter is included in the $G$ and one can view it a
``nuisance'' parameter. Robinson (1991) suggests REML estimation
as a way of ``estimating'' the smoothness parameter. Speed notes that
this is equivalent to Wahba's Generalized Maximum Likelihood estimate
of the smoothing parameter. 

Furthermore, this idea permits us to model nested curves in a natural
way. See Brumback and Rice (1998).

