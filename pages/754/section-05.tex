\chapter{Resampling methods: Bias, Variance, and their trade-off}
We have defined various smoothers and nonparametric estimation
techniques. In classical statistical theory we usually assume that the
underlying model generating the data is in the family of models we
are considering. For nonparametrics this assumptions is relaxed and
asymptotic and finite sample bias and variance 
estimates are not always easy to find in closed form. In this Chapter
we discuss some resampling methods that are commonly used to get
approximations of bias, variance, confidence intervals, etc... 

In particular we will look at the problem of choosing smoothing
parameters. Remember how most of the smoothers we have defined have
some parameter that controls the smoothness of the final smooth or
curve estimate. For kernel smoothers we defined the scale parameter, for
local regression we defined the span or bandwidth, and for smoothing
splines we had the penalty term. We will call all of these {\it
the  smoothing parameter} and denote it with $\lambda$. It should be
clear from the context which of the specific smoothing parameters we
are referring to.


\section{The bias-variance trade-off}

In smoothing in general there is a fundamental trade-off between the
bias and variance of the estimate, and this trade-off is governed by
the smoothing parameter. 

Through out this section we will be using an artifical example defined
by
\begin{equation}
\label{sinexample}
y_i = 5 \sin (1/x) + \epsilon_i, i=1,\dots,n
\end{equation}
with the $\epsilon_i$ IID $N(0,1)$ or $t_3$. 

\begin{figure}
\centerline{\epsfig{figure=Plots/plot-05-01.ps,angle=270,width=.8\textwidth}}
\caption{Outcomes of model with $f(x) = 5\sin(1/x)$ and IID normal
  errors with $\sigma^2=1$}
\end{figure}

The trade-off is most easily seen in the case of the running mean
smoother. The fitted running-mean smooth can be written as
\[
\hat{f}_k(x_0) = \frac{1}{2k+1} \sum_{i \in N^S_k(x_0)} y_i
\]
Under model (\ref{simplemodel}). The variance is easy to
compute. What is it? 

The bias is
\[
\E[\hat{f}_k(x_0)] -f(x_0) =  \frac{1}{2k+1} \sum_{i \in N^S_k(x_0)}
\{ f(x_i) - f(x_0)   \}
\]


Notice that as $k$, in this case the smoothing parameter, grows the
variances decreases. However, the bigger the $k$ the more $f(x_i)$'s
get into the bias.

We have no idea of what $\sum_{i \in N^S_k(x_0)} f(x_i)$ is
because we don't know $f$! Let's see this in a more precise (not
much more) way. 


Say we think that $f$ is smooth enough for us to assume that its second
derivative $f''(x_0)$ is bounded. Taylor's theorem says we can write 
\[
f(x_i) = f(x_0) + f'(x_0) (x_i - x_0) + \frac{1}{2}f''(x_0)(x_i-x_0)^2 +
o(|x_i-x_0|^2).
\]
Because $ \frac{1}{2}f''(x_0)(x_i-x_0)^2$ is $O(|x_i-x_0|^2)$ we stop
being precise and write 
\[
f(x_i) \approx f(x_0) + f'(x_0) (x_i - x_0) + \frac{1}{2}f''(x_0)(x_i-x_0)^2.
\]
Implicit here is the assumption that $|x_i-x_0|$ is small. This is the
way these asymptotics work. We assume that the kernel size
goes to 0 as $n$ gets big. 


Why did we only go up to the second derivative?

To makes things simple, let's assume that the covariates $x$ are {\it equally spaced} and
let $\Delta = x_{j+1}-x_j$ we can write 
\[
(2k+1)^{-1}\sum_{i \in N^S_k(x_0)} f(x_i) \approx f(x_0) + (2k+1)^{-1}\frac{k(k+1)}{6} f''(x_0) \Delta^2
\]
So now we see that the bias increases with $k^2$ and the second
derivative of the ``true'' function $f$. This agrees with our
intuition.
 
Now that we have
\[
\E\{\hat{f}_k(x_0) - f(x_0)\}^2 \approx \frac{\sigma^2}{2k+1} + \frac{k(k+1)}{6} f''(x_0) \Delta^2
\]
we can actually find an optimal $k$
\[
k_{opt} = \left\{ \frac{9\sigma^2}{2\Delta^4\{f''(x_i)\}^2}\right\}
\]
Usually this is not useful in practice because we have no idea of what
$f''(x)$ is like. So how do we chose smoothing parameters?

In Figure \ref{f5.1.2} we show the smooths obtained with a running
mean smoother with bandwidths of 0.01 and 0.1 on 25 replicates defined
by (\ref{sinexample}). The bias-variance trade-off can be clearly seen.

\begin{figure}
\centerline{\epsfig{figure=Plots/plot-05-02.ps,angle=270,width=.8\textwidth}}
\caption{\label{f5.1.2}Smooths using running-mean smoother with bandwidths of .01
  and 0.1. To the right are the smooths 25 replicates}
\end{figure}


\subsection{Bias-variance trade-off for linear smoothers}
Define $\bS_{\lambda}$ as the hat matrix for a particular smoother when
the smoothing parameter $\lambda$ is used. The ``smooth'' will be
written as $\hat{\f}_{\lambda} = \bS_{\lambda} \by$. 

Define 
\[
\bv_{\lambda} = \f - \E(\bS_{\lambda}\by)
\] 
as the {\it bias} vector.

Define $\ave(\bx^2) = n^{-1} \sum_{i=1}^n x_i^2$ for any vector
$\bx$. We can derive the following formulas:
\begin{eqnarray*}
\mbox{MSE}(\lambda) &=& n^{-1} \sum_{i=1}^n
\var\{\hat{f}_{\lambda}(x_i)\} + \ave ( \bv_{\lambda}^2 ) \\
&=& n^{-1}\tr(\bS_{\lambda}\bS_{\lambda}') \sigma^2 + n^{-1}
\bv_{\lambda}'\bv_{\lambda}\\
\mbox{PSE}(\lambda) &=& \{1 + n^{-1} \tr(\bS_{\lambda}\bS_{\lambda}')
\} \sigma^2 +  n^{-1}
\bv_{\lambda}'\bv_{\lambda}.
\end{eqnarray*}

Notice for least-squares regression $\bS_{\lambda}$ is idempotent so
that $\tr(\bS_{\lambda}\bS_{\lambda}') = \tr(\bS_{\lambda}) =
\mbox{rank}(\bS_{\lambda})$ which is usually the
number of parameters in the model. This is why we will sometimes refer
to $\tr(\bS_{\lambda}\bS_{\lambda}')$ as the {\it equivalent number of
  parameters} or degrees of freedom of our smoother.

\section{Cross Validation: Choosing smoothness parameters}
In the section, and the rest of the class, we will denote with
$\hat{f}_{\lambda}$ the
estimate obtained using smoothing parameter
$\lambda$. Notice that usually what we really have is the smooth
$\hat{\f}_{\lambda}$. 

We will use the model defined by (\ref{sinexample}). Figure \ref{f5.2.1}
shows one outcome of this model with normal and t-distributed errors.

\begin{figure}
\begin{center}
\epsfig{figure=Plots/plot-05-03.ps,angle=270,width=.8\textwidth}
\end{center}
\caption{\label{f5.2.1} Outcomes of model (\ref{sinexample})}
\end{figure}

We are trying to find the $\lambda$ that minimizes 
\[
\mbox{MSE}(\lambda) = n^{-1} \sum_{i=1}^n \E [ \hat{f}_{\lambda}(x_i) - f(x_i) ]^2
\]
Problem is we don't now $f$. 

What if we could get a new set of data $y^*_1,\dots,y^*_n$ from the
same model producing the $y_1,\dots,y_n$? This
would be quite helpful because the {\it predictive squared error} 
\[
\mbox{PSE}(\lambda)=\E[ y^*_i - \hat{f}_{\lambda}(x_i) ]^2 = \E[ \{y^*_i - f(x_i)\} -
\{\hat{f}_{\lambda}(x_i) - f(x_i)\} ] = \mbox{MSE}(\lambda) + \sigma^2.
\]
says that $n^{-1}\sum_{i=1}^n [ y^*_i - \hat{f}_{\lambda}(x_i) ]^2$ is 
an average  having
expected value the MSE plus a constant. We could view this quantity as
an estimate of $\mbox{MSE}(\lambda) + \sigma^2$. Since $\sigma^2$
doesn't depend on $\lambda$ we could find the $\lambda$ that minimizes
it and think that we are close to the 
$\lambda$ that minimizes the MSE. 

Notice that the above calculation can be done because the $y^*_i$s are
independent of the estimates 
$\hat{f}_{\lambda}(x_i)$s, the same can't be said about the $y_i$s. 

In practice it is not common to have a new set of data
$y_i^*,i=1,\dots,n$. Cross-validation tries to imitate this by
leaving out points
$(x_i,y_i)$ one at a time and estimating the smooth at $x_i$ based
on the remaining $n-1$ points. The cross-validation
sum of squares is
\[
\mbox{CV}(\lambda) = n^{-1} \sum_{i=1}^n \{ y_i -
\hat{f}_{\lambda}^{-i}(x_i) \}^2
\]
where $\hat{f}_{\lambda}^{-i}(x_i)$ indicates the fit at $x_i$
computed by leaving out the $i-th$ point. 

We can now use CV to choose $\lambda$ by considering a wide span of values
of $\lambda$, computing CV($\lambda$) for each one, and choosing the
$\lambda$ that minimizes it. Plots of CV($\lambda$) vs. $\lambda$ may
be useful. 

Why do we think this is good? First notice that 
\begin{eqnarray*}
\E\{y_i  - \hat{f}_{\lambda}^{-i}(x_i) \}^2 &=& \E \{ y_i - f(x_i) +
f(x_i) - \hat{f}_{\lambda}^{-i}(x_i) \}^2\\
&=& \sigma^2 + \E\{\hat{f}_{\lambda}^{-i}(x_i) - f(x_i)\}^2.
\end{eqnarray*}
Using the assumption that $\hat{f}_{\lambda}^{-i}(x_i) \approx
\hat{f}_{\lambda}(x_i)$ we see that
\[
\E\{\mbox{CV}(\lambda)\} \approx \mbox{PSE}(\lambda)
\]
However, what we really want is
\[
\min_{\lambda} \E\{\mbox{CV}(\lambda)\} \approx \min_{\lambda} \mbox{PSE}(\lambda)
\]
but the law of large numbers says the above will do.

Why not simply use the averaged squared residuals 
\[
\mbox{ASR}(\lambda)= n^{-1} \sum_{i=1}^n \{y_i - \hat{f}_{\lambda}(x_i)\}^2?
\]
It turns out this under-estimates the PSE. Notice in particular that 
the estimate $\hat{f}(x_i) = y_i$ always has ASR equal to 0! We will
see how we can adjust the ASR to form ``good'' estimates of the MSE. 

\subsection{CV for linear smoothers}
Now we will see some of the practical advantages of linear smoothers. 

For linear smoothers in general it is not obvious what is meant by
$\hat{f}_{\lambda}^{-i}(x_i)$. Let's give a definition...

Notice that any reasonable smoother will smooth constants into
constants, i.e. $\bS {\mathbf 1} = {\mathbf 1}$. If we think
of the rows $\bS_{i\cdot}$ of $\bS$ as weights of a kernels,
this condition is 
requiring that all the $n$ weights in each of the $n$ kernels add up
to 1. We can define 
$\hat{f}_{\lambda}^{-i}(x_i)$ as the ``weighted average'' 
\[
\bS_{i\cdot} \by = \sum_{j=1}^n S_{ij} y_j
\]
but giving zero weight to the $i$th entry, i.e.
\[
\hat{f}_{\lambda}^{-i}(x_i) = \frac{1}{1 - S_{ii}}\sum_{j\neq i} S_{ij}
y_j.
\]
From this definition we can find CV without actually making all the
computations again. Lets see how:

Notice that
\[
\hat{f}_{\lambda}^{-i}(x_i) = \sum_{j\neq i} S_{ij} y_j + S_{ii} 
\hat{f}_{\lambda}^{-i}(x_i).
\]
The quantities we add up to obtain CV are the squares of
\[
y_i - \hat{f}_{\lambda}^{-i}(x_i) = y_i -  \sum_{j\neq i} S_{ij} y_j -
S_{ii}  \hat{f}_{\lambda}^{-i}(x_i).
\]
Adding and subtracting $S_{ii} y_i$ we get
\[
y_i - \hat{f}_{\lambda}^{-i}(x_i) = y_i - \hat{f}_{\lambda}(x_i) + S_{ii}
( y_i - \hat{f}_{\lambda}^{-i}(x_i))
\]
which implies
\[
y_i - \hat{f}_{\lambda}^{-i}(x_i) = \frac{ y_i -
  \hat{f}_{\lambda}(x_i)}{1 - S_{ii}}
\]
and we can write
\[
\mbox{CV}(\lambda) = n^{-1} \sum_{i=1}^n  \left\{ \frac{ y_i -
  \hat{f}_{\lambda}(x_i)}{1 - S_{ii}} \right\}^2
\]
so we don't have to compute $\hat{f}_{\lambda}^{-i}(x_i)$!

Lets see how this definition of CV may be useful in finding the MSE.

Notice that the above defined CV is similar to the ASR except for
the division by $1 - S_{ii}$. To see what this is doing we notice that
in many situations $S_{ii} \approx [\bS_{\lambda} \bS_{\lambda}]_{ii}$
and $1/(1 - S_{ii})^2 \approx 1 + 2S_{ii}$ which implies
\[
\E[\mbox{CV}(\lambda)] \approx \mbox{PSE}(\lambda) + 2
\ave[\mbox{diag}(\bS_{\lambda}) \bv^2].
\]

Thus CV adjusts ASR so that in expectation the variance term is
correct but in doing so induces an error of $2S_{ii}$ into each of the
bias components.

In Figure \ref{f5.2.2} we see the CV and MSE for $n=100$ and $n=500$
observatios 
\begin{figure}
\begin{tabular}{c}
\epsfig{figure=Plots/plot-05-04.ps,angle=270,width=.8\textwidth}\\
\epsfig{figure=Plots/plot-05-05.ps,angle=270,width=.8\textwidth}
\end{tabular}
\caption{\label{f5.2.2}CV, MSE, and fits obtained for the normal and t
  models.}
\end{figure}


\section{Bootstrap Standard Errors and Confidence Sets }
Statistical science is the science of learning from experience. 
Efron and Tibshirani (1993) say ``Most people are not natural-born
statisticians. Left to our own devices we are not very good at picking
out patterns from a sea of noisy data. To put it another way, we are
all too good at picking out non existing patterns that happen to suit
our purposes.''

%%%from efron and tibshi
Suppose we find ourselves in the following common data-analytic
situation: a random sample $\bx = (x_1,\dots,x_n)$ from an unknown
probability distribution $F$ has been observed and we wish to estimate
a parameter of interest $\theta = t(F)$ on the basis of $\bx$. For
this purpose, we calculate an estimate $\hat{\theta} = s(\bx)$ from
$\bx$. 

A common estimate is the {\it plug-in} estimate $t(\hat{F})$
where $\hat{F}$ is the empirical distribution defined by
\[
F(x) = \frac{\mbox{ number of values in } \bx \mbox{ equal
    to }x}{n} 
\]
Can you think of a plug-in estimate that is commonly used?


The bootstrap was introduced by Efron (1979) as a computer based
method to estimate the standard deviation of $\hat{\theta}$. 

What are the advantages:
\begin{itemize}
\item It is completely automatic
\item Requires no theoretical calculations
\item Not based on asymptotic results
\item Available no matter how complicated the estimator $\hat{\theta}$
  is.
\end{itemize}

A bootstrap sample is defined to be a random sample of size $n$ drawn
from $\hat{F}$, say $\bx^* = (x_1^*,\dots,x_n^*)$.

For each bootstrap sample $\bx^*$ there is a bootstrap replicate of
$\hat{\theta}$, 
\[
\hat{\theta}^* = s(\bx^*).
\]
The bootstrap estimate of $\se_F(\hat{\theta})$ is defined by 
\begin{equation}
\label{ibe}
\se_{\hat{F}}(\hat{\theta}^*).
\end{equation}
This is called the {\it ideal bootstrap estimate} of the standard
error of $s(\bx)$.  

Notice that for the case where $\theta$ is the
expected value or mean of $\bx_1$ we have
\[
\se_{\hat{F}}(\bar{x}^*) = \se_{\hat{F}}(x^*_1)/\sqrt{n} =
\sqrt{n^{-1} \sum_{i=1}^n (x_i - \hat{x})^2}/\sqrt{n}
\]
and the ideal bootstrap estimate is the estimate we are used
to. However, for any other estimator other than the mean obtaining
(\ref{ibe}) there is no neat formula that enables us to compute a
numerical value in practice.

The bootstrap algorithm is a computational way of obtaining a good
approximation to the numerical value of (\ref{ibe}).

\subsection{The bootstrap algorithm}
The bootstrap algorithm works by drawing many independent bootstrap
samples, evaluating the corresponding bootstrap replications, and
estimating the standard error of $\hat{\theta}$ by the empirical standard
error, denoted by $\hat{\se}_B$, where $B$ is the number of bootstrap
samples used.

\begin{enumerate}
\item Select $B$ independent bootstrap samples
  $\bx_1^*,\dots,\bx^*_B$, each consisting of $n$ data values drawing
  with replacement from $\bx$. 
\item Evaluate the bootstrap replication corresponding to each
  bootstrap sample
\[
\hat{\theta}^*(b) = s(\bx_b^*), b=1,\dots,B
\]
\item Estimate the standard error $\se_{F}(\hat{\theta})$ by the
  sample standard error of the $B$ replicates
\[
\hat{\se}_B = \left[ \frac{1}{B-1}\sum_{b=1}^B \{ \hat{\theta}^*(b) -
  \hat{\theta}^*(\cdot) \}^2 \right]
\]
with
\[
\hat{\theta}^*(\cdot) = B^{-1} \sum_{b=1}^B  \hat{\theta}^*(b)
\]
\end{enumerate}
The limit of $\hat{\se}_B$ as $B$ goes to infinity is the ideal
bootstrap estimate of (\ref{ibe}). But how close is (\ref{ibe}) to
$\se_F(\hat{\theta})$? See Efron and Tibshirani (1993) for more details.


\subsection{Example: Curve fitting}
In this example we will be estimating regression functions in two
ways, by a standard least-squares line and by loess. 
\begin{figure}
\begin{center}
\epsfig{figure=Plots/plot-05-06.ps,angle=270,width=.8\textwidth}
\end{center}
\caption{\label{f5.3.1} Estimated regression curves of Improvement on
  Compliance.}
\end{figure}


A total of 164 mean took part in an experiment to see if the drug
cholostyramine lowered blood cholesterol levels. The men were supposed
to take six packets of cholostyramine per day, but many of them
actually took much less. Figure \ref{f5.3.1} shows compliance plotted
against percentage of the intended dose actually taken. We also show a
fitted line and a loess fit (using span=2/3). Notice the curves
similar from 0 to 60, a little different from 60 to 80 and quite
different from 80 to 100. 

Assume the points a regression model 
\[
y_i = f(x_i) + \varepsilon_i, i=1,\dots,n
\]
with the $\varepsilon_i$ IID. 

Say we are interested in the difference in rate of change of $f(x)$ in
the 60--80 and 80--100
sections. We could define as the parameter to describe this.
How can we do this?

Notice that finding a standard error for this estimate is not
straight-forward. We can use the bootstrap. 

\begin{table}[htb]
\caption{Estimates and bootstrap standard errors of $f(60), f(80),$
  and $f(100)$.}
\begin{tabular}{lcccccc}
 &$\hat{f}_{\mbox{line}}(60)$&$\hat{f}_{\mbox{line}}(80)$&$
 \hat{f}_{\mbox{line}}(100)$&$\hat{f}_{\mbox{loess}}(60)$&$
   \hat{f}_{\mbox{loess}}(80)$&$\hat{f}_{\mbox{loess}}(100)$\\
\hline
value:&33&44&56&28&35&66\\
$\hat{\se}_{50}$:&2&2&3&5&4&4
\end{tabular}
\end{table}

As seen in Figure \ref{f5.3.2}. Even when there is no parameter of interest, the bootstrap estimates
of $f$ give us an idea of what a confidence set is for the
nonparametric estimates. We will see more of this in Chapter 7 and 8.

\begin{figure}
\begin{center}
\epsfig{figure=Plots/plot-05-07.ps,angle=270,width=.8\textwidth}
\end{center}
\caption{\label{f5.3.2} 50 boostrap curves for each estimation techinique.}
\end{figure}

\subsection{Confidence ``intervals'' for linear smoothers}
It is easy to show that the variance-covariance matrix of the vector
of fitted values $\hat{\f} = \bS \by$ is
\[
\cov(\hat{\f}) = \bS \bS' \sigma^2
\]
and given an estimate of $\sigma^2$ this can be used to give
point-wise standard errors, mainly by looking at
$\mbox{diag}(\bS\bS')\sigma^2$.  

Can we construct confidence intervals? What do we need?

First of all we need to know the distribution (at least approximately)
of $\hat{\f}$. If the errors are normal we know that $\bf{\f}$ is normally
distributed. Why?

In the normal case, what are the confidence intervals for? 

Remember that our estimates are usually biased, $\E(\hat{\f}) = \bS \f
\neq \f$. If our null hypothesis is $\bS \f = \f$ (in the case of splines
this is equivalent to assuming $f \in \cal G$) then our confidence
intervals are for $\f$ otherwise it is much more convinient to compute
them for $\bS \f$. We will start
using the notation $\tilde{\f} = \bS \f$. We can think of $\tilde{\f}$
as the best possilble approximation to ``the truth'' $\f$ when using
the $\bS$ as a smoother.

To see how point-wise estimates can be useful, notice that we can get
an idea of how variable $\hat{\f}(x_0)$ is. However, it isn't very
helpful when we want to see how variable $\hat{\f}$ is as a whole.

What if we want to know if a certain function, say a line, is
in our ``confidence interval''? Point-wise intervals don't really help
us with this.

\subsection{Global confidence bands}
Remember that $\hat{\f} \in {\mathbb R}^n$. This means that 
talking about confidence intervals doesn't make much sense. We need to
consider confidence sets. 

For example if the errors are normal we know that 
\[
\chi(\tilde{\f}) = (\hat{\f} - \tilde{\f})'(\bS \bS'
\sigma^2)^{-1}(\hat{\f} - \tilde{\f}) 
\]
is $\chi^2_n$ distributed. This permits us to construct confidence
sets (which you can think of as random $n$-dimensional balls) 
for $\tilde{\f}$ of probability $\alpha$
\[
C_{\alpha} = \{ \g \in {\mathbb R}^b ; \chi(\g) \leq \chi_{1-\alpha}
\}=\{ \g  \in {\mathbb R}^b; (\hat{\f}
- \g)'(\bS \bS' \sigma^2)^{-1}(\hat{\f} - \g) \leq \chi_{1-\alpha} \}.
\]
Notice that the probability that the random ball doesn't fall on the
approximate truth $\tilde{\f}$ is $\alpha$:
\[
\Pr(\tilde{\f} \not\in C_{\alpha}) = \Pr\left[(\hat{\f} - \tilde{\f})'(\bS \bS'
\sigma^2)^{-1}(\hat{\f} - \tilde{\f}) > \chi_{1-\alpha}\right]= \alpha.
\]
This is only the case if we know $\sigma^2$. 

Usually we construct an estimate 
\[
\hat{\sigma}^2 = (\by - \hat{\f})' (\by - \hat{\f})/ \{n - \tr(2\bS -
\bS\bS')\}
\]
and define confidence sets 
\[
C(\tilde{\f}) = \{ \g  \in {\mathbb R}^b; \nu(\g) \leq G_{1-\alpha} \}
\]
based on 
\[ 
\nu(\tilde{\f}) =  (\hat{\f} - \tilde{\f})'(\bS \bS'
\hat{\sigma}^2)^{-1}(\hat{\f} - \tilde{\f}). 
\]
Here $G_{1-\alpha}$ is the $(1-\alpha)$th quantile of the distribution
of $\nu(\tilde{\f})$.

Do we know $G$? Not necessarily.

In the case of linear regression, where the gaussian model is correct
and $\bS$ is a 
p-dimensional projection, $\nu(\tilde{\f})=\nu(\f)$ has distribution
$(n-p) + p F_{p,n-p}$. 

When this is not the case we can argue that the
distribution is approximately 
\[
\{n-\tr(2\bS - \bS\bS')\} + \tr(\bS\bS') F_{\tr(\bS\bS'),n-\tr(2\bS -
  \bS\bS')}
\]

If we are not sure of the normality assumption or that $\tilde{\f}
\approx \f$ we can use the bootstrap to construct an approximate
distribution $\hat{G}$ of $G$.

How do we do it?

\begin{figure}[htb]
\caption{The regression curve and an outcome with $n=100$ and
  $\sigma^2=1$.}
\begin{center}
\epsfig{figure=Plots/plot-05-08.ps,angle=270,width=.8\textwidth}
\end{center}
\end{figure}

\subsection{Bootstrap estimate of $G_{1-\alpha}$}
A bootstrap sample is generated in the following way
\begin{itemize}
\item For some data $\by$ use some procedure (a linear smoother for
  example) to obtain an estimate $\hat{\f}$ of some estimand (in this
  case the regression function $\f$). 
\item Obtain residuals $\hat{\bg{\varepsilon}} = \by - \hat{\f}$. 
\item Take a simple random sample of size $B$ from the residuals
  $\hat{\varepsilon}_1,\dots,\hat{\varepsilon}_n$. Notice that this makes
  them IID just like the $\varepsilon$s.
\item Construct a ``new'' data set 
\[\by^* = \hat{\f} +  \hat{\bg{\varepsilon}}^*
\]
with $\hat{\bg{\varepsilon}}^*$ the vector of resampled residuals.
\item From the new data form a new estimate $\hat{\f}^*$.
\item Finally we obtain the value of 
\[
\nu^* = (\hat{\f}^* - \hat{\f})'(\bS \bS'
\hat{\sigma}^{*2})^{-1}(\hat{\f}^* - \hat{\f})
\]
\item We repeat this procedure many times and form an approximate
  distribution $\hat{G}$ with the values of $\nu^*$. We may use the $(1-\alpha)$th
  quantile of $\hat{G}$ as an estimate of $G_{1-\alpha}$.
\end{itemize}


Let's consider the model $y_i = f(x_i) + \varepsilon_i, i=1,\dots,n$
with $\varepsilon_i$ IID normal.  In Figure \ref{f5.8} we see
qqplots of the true $G$, the bootstrap $G$ and the F-distribution
approximation. 




\subsection{Displaying the confidence sets}
Displaying an $n-dimensional$ ball is not easy. 

Global confidence
bands usually show the projections of the confidence set onto each of
the component sub-spaces. Notice that a
function (now I'm using function and $n$-dimensional vector
interchangeably) in this set would actually be in a confidence cube as opposed
to a ball! So a vector within the confidence bands isn't necessarily in the confidence
ball. However its true that being in the ball implies being within the
band. 

Another popular approach is selecting a few functions at random from
$N(\hat{\f},\bS\bS'\hat{\sigma}^2)$ and checking to see if they are in
the confidence set. If they are, we plot them. This enables us to see
what kind of ``shape'' functions in the confidence set have. Maybe
they all have a bump, maybe a large amount of them are close to being
constant lines, etc...

\begin{figure}[htb]
\caption{\label{f5.8} QQ-plot of bootstrap vs. true $G$ and the
  F-distribution approximation. We also see point-wise confidence
  intervals and curves in (blue) and out (green) of the bootstrap
  confidence set.}
\begin{center}
\epsfig{figure=Plots/plot-05-09.ps,angle=270,width=.8\textwidth}
\end{center}
\end{figure}

\subsection{Approximate F-test}
Using the F-distribution approximations we may construct F-tests for 
testing various hypotheses.

The p-value given by the S-Plus function {\tt gam()} is usually
testing for linearity and using an F-distribution approximation.

Suppose we wish to compare 2 smoothers $\hat{\f}_1 = \bS_1 \by$ and
$\hat{\f}_2 = \bS_2 \by$. For example, $\hat{\f}_1$ may be linear
regression and $\hat{\f}_2$ may be a ``rougher'' smoother. 

Let $RSS_1$ and $RSS_2$ be the residual sum of squares obtained for
each smoother. Which one do you expect to be bigger?
 
and $\gamma_1$ and $\gamma_2$ be the degrees of freedom of each
smoother, $\tr(2\bS_j - \bS_j\bS_j'), j=1,2$. An approximation that
may be useful for this comparison is
\[
\frac{(RSS_1 - RSS_2)/(\gamma_2 - \gamma_1)}{RSS2/(n-\gamma_2)} \sim
F_{\gamma_2-\gamma_1,n-\gamma_2}
\]

There are moment corrections that can make this a better
approximation (see H\&T).
\begin{figure}[htb]
\caption{Same as previos figure but with t-distributed errors}
\begin{center}
\epsfig{figure=Plots/plot-05-10.ps,angle=270,width=.8\textwidth}
\end{center}
\end{figure}